\documentclass[a4paper,12pt]{article}

\usepackage[margin=2.5cm]{geometry}
\usepackage[esperanto]{babel}
\usepackage[T1]{fontenc}
\usepackage[utf8]{inputenc}
\usepackage{lmodern}
\usepackage{tikz}
\usepackage{amsmath}
\usepackage{caption}

\usetikzlibrary{arrows.meta,calc}

\title{Optikaj fenomenoj de lumo}
\author{}
\date{}

\begin{document}
\maketitle

\section*{Reflekto de la lumo}

\begin{figure}[h!]
\centering
\begin{tikzpicture}[scale=1.1,>=Latex]

\draw[thick] (-4,0) -- (4,0);
\node[below] at (0,-0.8) {Spegula surfaco};

\draw[dashed,gray] (0,-2) -- (0,2);
\node[above] at (0,1.8) {Normala linio};

\draw[very thick,red,-{Latex[length=3mm]}] (-3,2) -- (0,0);
\node[above left] at (-1.5,1.3) {Enira luma radiuso};

\draw[very thick,blue,-{Latex[length=3mm]}] (0,0) -- (3,2);
\node[above right] at (2.5,1.3) {Reflektita luma radiuso};

\fill (0,0) circle (1.5pt);
\node[below right] at (0.1,0) {Reflekta punkto};

\end{tikzpicture}
\caption{Reflekto de la lumo sur spegula surfaco}
\end{figure}

\section*{Refrakto de la lumo}

\begin{figure}[h!]
\centering
\begin{tikzpicture}[scale=1.1,>=Latex]

\fill[blue!10] (-4,0) rectangle (4,-2);
\node[left] at (-1,-1.5) {Densa medio (ekzemple akvo)};
\node[above] at (0,2.2) {Malpli densa medio (ekzemple aero)};

\draw[thick] (-4,0) -- (4,0);

\draw[dashed,gray] (0,-2) -- (0,2);
\node[above] at (0,1.8) {Normala linio};

\draw[very thick,red,-{Latex[length=3mm]}] (-3,2) -- (0,0);
\node[above left] at (-2.5,1.3) {Enira luma radiuso};

\draw[very thick,green!60!black,-{Latex[length=3mm]}] (0,0) -- (2,-2);
\node[right] at (1.7,-1.4) {Refraktita luma radiuso};

\fill (0,0) circle (1.5pt);
\node[above right] at (0.1,0.05) {Transira punkto};

\end{tikzpicture}
\caption{Refrakto de la lumo ĉe limo inter du medioj}
\end{figure}

\end{document}

